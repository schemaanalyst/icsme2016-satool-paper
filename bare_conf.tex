\documentclass[conference]{IEEEtran}

% Packages that Cody added start here
\usepackage{tikz}
\usetikzlibrary{shapes,arrows,shadows,positioning,calc}
\usepackage{listings}
\usepackage{textcomp}
\usepackage[T1]{fontenc}
\lstset{basicstyle=\small\ttfamily,
columns=flexible,
breaklines=true,
upquote=true
}
\usepackage{hhline}
% Packages that Cody added end here

\begin{document}

\title{\textit{SchemaAnalyst}: Search-based Test Data Generation for Relational Database Schemas}


% author names and affiliations
% use a multiple column layout for up to three different
% affiliations
\author{
\IEEEauthorblockN{Phil McMinn and Chris Wright}
\IEEEauthorblockA{University of Sheffield}
\and
\IEEEauthorblockN{Cody Kinneer, Colton McCurdy, Michael Camara\\ and Gregory M. Kapfhammer}
\IEEEauthorblockA{Allegheny College}}

% conference papers do not typically use \thanks and this command
% is locked out in conference mode. If really needed, such as for
% the acknowledgment of grants, issue a \IEEEoverridecommandlockouts
% after \documentclass

% for over three affiliations, or if they all won't fit within the width
% of the page, use this alternative format:
% 
%\author{\IEEEauthorblockN{Michael Shell\IEEEauthorrefmark{1},
%Homer Simpson\IEEEauthorrefmark{2},
%James Kirk\IEEEauthorrefmark{3}, 
%Montgomery Scott\IEEEauthorrefmark{3} and
%Eldon Tyrell\IEEEauthorrefmark{4}}
%\IEEEauthorblockA{\IEEEauthorrefmark{1}School of Electrical and Computer Engineering\\
%Georgia Institute of Technology,
%Atlanta, Georgia 30332--0250\\ Email: see http://www.michaelshell.org/contact.html}
%\IEEEauthorblockA{\IEEEauthorrefmark{2}Twentieth Century Fox, Springfield, USA\\
%Email: homer@thesimpsons.com}
%\IEEEauthorblockA{\IEEEauthorrefmark{3}Starfleet Academy, San Francisco, California 96678-2391\\
%Telephone: (800) 555--1212, Fax: (888) 555--1212}
%\IEEEauthorblockA{\IEEEauthorrefmark{4}Tyrell Inc., 123 Replicant Street, Los Angeles, California 90210--4321}}




% use for special paper notices
%\IEEEspecialpapernotice{(Invited Paper)}




% make the title area
\maketitle

% As a general rule, do not put math, special symbols or citations
% in the abstract
\begin{abstract}
Data stored in relational databases plays a vital role in many aspects of life.
When this data in incorrect, the services that depend on that data may be compromised.
The database schema is the artefact responsible for maintaining the integrity of data. Because of this
critical function, properly testing the database schema is a task of great importance.
\textit{SchemaAnalyst} is a tool to support testing this vital software component.
\textit{SchemaAnalyst} uses a search-based approach to generate high-quality test data for database
schemas. \textit{SchemaAnalyst} is extensible, includes a comprehensive evaluation framework for
evaluating the quality of test data, and includes documentation. This paper presents the release of
\textit{SchemaAnalyst} to enhance the ability of practitioners to test database schemas.
\end{abstract}

% no keywords




% For peer review papers, you can put extra information on the cover
% page as needed:
% \ifCLASSOPTIONpeerreview
% \begin{center} \bfseries EDICS Category: 3-BBND \end{center}
% \fi
%
% For peerreview papers, this IEEEtran command inserts a page break and
% creates the second title. It will be ignored for other modes.
\IEEEpeerreviewmaketitle

\section{Introduction}\label{sec:intro}

% Introduction to the importance of databases

Healthcare, science, and commerce often rely on information that is stored in
databases~\cite{kapfhammer2007comprehensive}.  When this data is incorrect, passengers can have their flights delayed or
patients may receive the wrong medication~\cite{databasebook}.  In addition to documenting the structure of and
connections between data entries, relational databases furnish a means for protecting the correctness of the data that
they store.  In particular, the relational database schema is the artifact that is responsible for safeguarding the
integrity of a relational database. The vital role of the database schema makes the testing of it a task of vital
importance.

% Comments on NoSQL systems and then explain why the relational DBMS is still important

While non-relational ``NoSQL'' database systems have been gaining in popularity, relational databases remain pervasive.
For instance, Skype, the widely used video-call software, uses the PostgreSQL database management system
(DBMS)~\cite{postgres} while Google makes use of the SQLite DBMS in Android phones~\cite{sqlite}.  Additionally,
according to DB-Engines.com, the top three most popular DBMSs are relational in nature~\cite{dbrank}; also, the 968,277
questions asked on StackExchange about relational databases show the demand for their support~\cite{stackexchange}.

% Explain more about the tool and cite some of its benefits

% GMK NOTE: The phrase "enables verifying" is awkward, but difficult to rephrase

\sa~is a tool for generating high-quality test data in support of database schema testing. Using a search-based approach
that evaluates fitness to incrementally improve test data~\cite{Korel:AVM}, {\sa} discovers test data instances that
comprehensively exercise the database schema.  {\sa} includes an evaluation framework with a collection of real-world
case study schemas, as well as a mutation analysis system that enables verifying the quality of the generated test data
based on its capability to detect systematically seeded faults.  Additionally, {\sa} is extensible, well documented, and
available on GitHub under an open-source license~\cite{tool}.

% Summarize when the tool has been used and comment on the benefits of using it

{\sa} has been used to support research studies focusing on both search-based software
testing~\cite{kapfhammer2013search,mcminn2015effectiveness,kinneer2015automatically} and mutation
testing~\cite{wright2013efficient,wright2014impact,wright2015mutation,mcminn2016virtual}.  In addition to describing the
implementation of \sa~and overviewing its efficiency and effectiveness, this paper inaugurates the public release of
this testing tool. Since past studies have shown the benefits of using the presented open-source tool instead of
competing systems, this paper argues that \sa~is ready to enhance practitioners' testing of database schemas.  In
summary, the key contributions of this paper are as follows:

% Key Contributions:

\begin{itemize}

  % The SchemaAnalyst tool itself

  \item {\sa}, an extensible, efficient, and effective tool that generates test data for database schemas
    (Section~\ref{sec:technique}).

  % The surrounding framework that includes the schemas needed for empirical study and the mutation analysis

  \item In support of researchers, a comprehensive evaluation framework, including relational schemas suitable for
    further empirical study, and mutation analysis tools supporting the assessment of test data quality
    (Section~\ref{sec:technique}).

  % Documentation of the key features of the tool (e.g., the command lines)

  \item Aiding both researchers and practitioners, documentation explaining the features and usage of the tool
    (Section~\ref{sec:implementation}).

  % Due to space constraints, a brief overview of past results obtained when using SchemaAnalyst

  \item Confirming the scalability and applicability of \sa, a survey of prior empirical results
    (Section~\ref{sec:relatedwork}).

\end{itemize}

\section{Background}\label{sec:background}

% Explain the idea of software testing

Software testing, the process of running a software system to ensure that it functions as intended, is a vital part of
the software development lifecycle. If software does not meet users' expectations, then it contains a fault.  Developers
can check for these faults by running tests that give the program inputs and check for expected
outputs~\cite{ammann2008}.  If the software produces the expected output for the provided input, then this suggests that
it is functioning correctly.  Yet, if it does not perform as anticipated, then the tests may have found \mbox{a fault}.

% Introduce the idea of a test case and then talk about search-based test data generators

A collection of test cases is called a test suite. A test suite's effectiveness at finding faults is known as its
adequacy, which is assessed by a test suite adequacy criterion.  Writing high-quality test suites requires developers to
painstakingly consider the range of possible inputs --- as anticipated, this is often a challenging and time-consuming
process. Test data generation reduces the burden on a human tester by (semi-)automatically creating test case inputs.
The focus of this paper, search-based test data generators, use a fitness function to direct the method towards the
creation of high-quality test data~\cite{STVR:STVR294}. The fitness function evaluates the quality of the test data,
allowing the data generator to iteratively pursue higher quality inputs.

% Introduce the idea of mutation testing to assess the quality of a test suite

Mutation testing is an adequacy criterion that measures the effectiveness of a test suite by modifying artefact under
test to produce a ``mutant''. The change is meant to simulate an artificial fault, and the mutant artifact is expected
to result in different behavior from the original. The result of the test suite from the original and mutant artefacts
are compared. If the results are the same, then the test suite failed to detect the artificially seeded fault. If the
results are different however, then the test suite was able to differentiate the two artifacts, finding the simulated
fault, and the mutant is said to be ``killed''.  The mutation score is the number of mutants killed divided by the
number of total mutants.

%!TEX root=../icsme2016_tool_paper.tex
\begin{figure}[t]
\centering
\scalebox{0.8}{
\begin{tabular}{r|l|}
\hhline{~-}
1 & \texttt{CREATE TABLE Inventory}\\
2 & \texttt{(}\\
3 & \texttt{  id INT PRIMARY KEY,}\\
4 & \texttt{  product VARCHAR(50) UNIQUE,}\\
5 & \texttt{  quantity INT,}\\
6 & \texttt{  price DECIMAL(18,2)}\\
7 & \texttt{);}\\
\hhline{~-}
\end{tabular}
}
\caption{\label{fig:schema}The Inventory relational database schema}
\vspace*{-1em}
\end{figure}


A database is a collection of data. Databases are managed by applications called database management systems
(DBMSs)~\cite{databasebook}.  Relational databases are databases whose data entries can refer to one another. The
database schema is the artefact that lays out the structure of the database, which is organized into tables and columns.
The schema can also define integrity constraints, which are rules that data submitted to the database must meet to be
accepted. If the data violates an integrity constraint specified by the schema, then it is rejected as invalid.
Figure~\ref{fig:schema} shows a simple database schema for recording the number of products kept in an inventory. This
schema defines one table, called \texttt{Inventory}, with four columns.  The \texttt{id} column on line three is given
the \texttt{PRIMARY KEY} constraint, which means that data inserted into this column cannot be left missing or unknown,
and that the values in this column must be unique. If the \texttt{PRIMARY KEY} constraint was left out of the schema by
mistake, then multiple items could be entered with the same \texttt{id} value, which could result in the wrong item
being shipped to the customer in an industrial application.

\section{\textit{SchemaAnalyst}}

While verifying the accuracy of the database schema is critical for protecting data integrity, manually
creating test data is expensive and time consuming. \textit{SchemaAnalyst} uses a search-based
approach in order to generate test data automatically. Figure~\ref{fig:schemaanalyst} provides a
high-level overview of the \textit{SchemaAnalyst} system.
After being given an input schema, a coverage criterion first generates a collection of test 
requirements. These test requirements are the rules that
the test data must fulfill. A test requirement for the \texttt{Inventory} schema in
Figure~\ref{fig:schema} might be, ``the \texttt{PRIMARY KEY} constraint on line three must be
violated''. A coverage criterion provides a way to systematically generate test requirements. One
example coverage criterion is Integrity Constraint Coverage (ICC). This criterion generates two test
requirements for every integrity constraint in the schema, one requiring that the constraint is
satisfied, and one requiring that the constraint is violated. \textit{SchemaAnalyst} includes nine
coverage criteria, which are explained in detail in~\cite{mcminn2015effectiveness}.

\textit{SchemaAnalyst} then generates test data
to satisfy the test requirements using a test data generator. The default test data generator used by
\textit{SchemaAnalyst} is based on Korel's Alternating Variable Method (AVM)~\cite{Korel:AVM}.
This data generator uses a fitness function to evaluate how well the test data satisfies the
requirements, and guide the test data generator towards producing higher quality test data that
satisfies more of the requirements. The resulting test data is saved as a JUnit test suite.

\textit{SchemaAnalyst} also includes features to evaluate the quality of the generated test data. The
coverage of the generated test data is the percentage of test requirements satisfied by the test data. 
Test data quality can also be measured using the provided mutation testing tools. When executed in
mutation testing mode, \textit{SchemaAnalyst} will generate mutant database schemas and compare the
behavior of the test data on the original and mutant schemas. \textit{SchemaAnalyst} includes 14
different mutation operators that can be used to assess test suite quality.

%!TEX root=../icsme2016_tool_paper.tex
\newcommand{\mx}[1]{\mathbf{\bm{#1}}} % Matrix command
\newcommand{\vc}[1]{\mathbf{\bm{#1}}} % Vector command

% Define the layers to draw the diagram
\pgfdeclarelayer{background}
\pgfdeclarelayer{foreground}
\pgfsetlayers{background,main,foreground}

% Define block styles used later

\tikzstyle{sensor}=[draw, fill=black!10, text width=5em,
    text centered, minimum height=2.5em,drop shadow]
\tikzstyle{ann} = [above, text width=5em, text centered]
\tikzstyle{wa} = [sensor, text width=10em, fill=black!30,
    minimum height=6em, rounded corners, drop shadow]
\tikzstyle{sc} = [sensor, text width=13em, fill=red!30,
    minimum height=10em, rounded corners, drop shadow]

% Define distances for bordering
\def\blockdist{1.5}
\def\edgedist{2.5}

\begin{figure}[t]
  \centering

\begin{tikzpicture}[thick,scale=0.85, every node/.style={scale=0.85}]
    \node (wa) [wa]  {\textit{SchemaAnalyst}};
    \path (wa.west)+(-\blockdist,-1.25) node (asr1) [sensor] {Database Schema};

    \path (wa.west)+(-\blockdist,1.25) node (asr2)[sensor] {Coverage Criterion};
    \path (wa.west)+(-\blockdist,0.0) node (dots)[sensor] {Data Generator};

    % \path (dots.west)+(-2.65*\blockdist,-0.0) node (dataa) [sensor] {Data Generator};
    % \path (asr2.west)+(-2.65*\blockdist,-0.0) node (criteriona) [sensor] {Coverage Criterion};

    \path (wa.east)+(\blockdist,0) node (vote) [sensor] {Test Suite};

    \path [draw, ->] (asr1.east) -- node [above] {}
        (wa.200);
    \path [draw, ->] (asr2.east) -- node [above] {}
        (wa.160);
    \path [draw, ->] (dots.east) -- node [above] {}
        (wa.180);
    \path [draw, ->] (wa.east) -- node [above] {}
        (vote.west);

    % \path [draw, ->] (dataa.east) -- node [above] {}
    %     (dots.west);
    % \path [draw, ->] (criteriona.east) -- node [above] {}
    %     (asr2.west);
    % \path [draw, ->] (schemaa.east) -| node [above] {}
    %     (doubler.230);
    %

\end{tikzpicture}

\caption{\label{fig:schemaanalyst}The inputs and outputs of the \textit{SchemaAnalyst} tool\vspace*{-1.75in}}

\end{figure}


\begin{figure}
\lstinputlisting[basicstyle=\ttfamily\footnotesize]{figures/Usage.txt}
\caption{\label{fig:usage} The first section of the \sa~help menu.}
\vspace*{-.1in}
\end{figure}

\section{Implementation}\label{sec:implementation}
\subsection{Design}

% Introduce the design and implementation of the SchemaAnalyst tool; closing with a commentary about the fact that the
% chosen SQL parser is a commercial product that we cannot release on GitHub.

\sa~is implemented in the Java programming language.  Designed with extensibility in mind, the \sa~tool is divided into
$13$ packages, which this paper briefly overviews. The \texttt{sqlrepresentation} package provides an intermediate Java
representation of data structures in relational databases, fully modelling database tables, columns, expressions, data
types, integrity constraints, and other relevant entities. These objects enable \sa~to support multiple DBMSs (i.e.,
\sqlite, \postgres, and \hypersql), and, additionally, allow for the inclusion of new DBMSs. The tool also contains the
\texttt{sqlparser} package that wraps the General SQL Parser~\cite{generalsqlparser}, thus enabling the effective
conversion of a schema expressed in the Structured Query Language (SQL) to the tool's intermediate representation. As
this SQL parser is a commercial product, the open-source version of \sa~does not provide it for download on GitHub.
Therefore, users can experiment with \sa~by either testing the provided schemas or (automatically or manually)
converting a new schema to the tool's internal SQL representation.

% Go into additional details about some of the packages (breaking up an otherwise too-long paragraph)

% GMK NOTE: This paragraph talks about packages in the tool that are not inside of the main technical diagram
% It would be best if there was a statement of why that is the case --- but, I could not easily fit that here

The \texttt{testgeneration} package provides a representation of test suites and test cases, as well as test
requirements and the nine provided coverage criteria~\cite{mcminn2015effectiveness}.  The \texttt{data} package
furnishes the three provided test data generators, as well as various generic data-type representations for use during
test data generation~\cite{mcminn2015effectiveness}; Table~\ref{tab:args} summarizes the coverage criteria and data
generators furnished by the \sa~tool.

% Summarize the final packages in the tool; note that some of these are not part of the diagram

The \texttt{dbms} package provides support for three DBMSs, and includes the classes that enable interaction with an
installed DBMSs. The \texttt{sqlwriter} package furnishes support for creating SQL statements for use with DBMSs, and is
used with the \texttt{javawriter} package to encode the generated test data as a JUnit test suite.  The
\texttt{mutation} package provides the mutation analysis functionality, including the mutation
operators~\cite{wright2015mutation}, mutant equivalence and reduction features~\cite{wright2014impact}, and means for
quickly performing mutation analysis~\cite{mcminn2016virtual}.

\subsection{Usage}

% NOTE: The lstinline command must be on a full line and no line wrapping is allowed (will cause compiler error)

% Explain how the tool is released, downloaded, and built and setup for running

\sa~is publicly available on GitHub under an open-source license~\cite{tool}.
After cloning the Git repository, the project can be built using Gradle by
running the following command in the project's root directory:
\lstinline{./gradlew compile}. After the tool compiles, the user must set the
\lstinline{CLASSPATH} so that it contains \lstinline{build/classes/main},
\lstinline{build/lib/*}, \lstinline{lib/*} and the current working directory.

% command: \lstinline{export CLASSPATH="build/classes/main:build/lib/*:lib/*:."}.

% Talk about installing the databases and then make a reference to the documentation

Optionally, the user can install the \postgres, \sqlite, and \hypersql~DBMSs. Since \sqlite~does not require
configuration on the computer running \sa, it is currently the default option. Using the chosen DBMS, \sa~will run the
generated test suite. If the use of an actual DBMS is desired, the user should refer to online documentation for detailed
instructions~\cite{tool}. The tool also supports a ``virtual'' DBMS executor allowing SQL statements to be simulated.

\begin{table}[t]
\centering
\caption{Key Features Provided by \textit{SchemaAnalyst}.}
\label{tab:args}

\begin{tabular}{rr}
  \begin{minipage}{1.25in}

    \begin{tabular}{l}
    Coverage Criteria \\
    \midrule
      APC                                    \\
      ICC                                    \\
      AICC                                   \\
      CondAICC                               \\
      ClauseAICC                             \\
      UCC                                    \\
      AUCC                                   \\
      NCC                                    \\
      ANCC
    \end{tabular}

  \end{minipage} &

  \begin{minipage}{1.25in}

    \begin{tabular}{l}
    Data Generators \\
    \midrule
      AVM -- Random Restart  \\
      AVM -- Default Restart \\
      AICC                                   \\
      CondAICC                               \\
      ClauseAICC                             \\
      UCC                                    \\
      AUCC                                   \\
      NCC                                    \\
      ANCC
    \end{tabular}

  \end{minipage}
\end{tabular}
\vspace*{-.25in}

\end{table}




% Basically, explain the help menu and then the way to generate test data at the command line

Usage instructions for \sa~can be obtained by running \lstinline{java org.schemaanalyst.util.Go --help}.
Figure~\ref{fig:usage} shows a snippet of this menu.  As indicated by the help display, \sa~first expects options
indicating the desired schema, coverage criterion, data generator, and DBMS\@. Defaults are provided for all of these
options except for the schema option, which is required. The user must then give a command.  The two supported commands
are \lstinline{generation}, used to generate test data, and \lstinline{mutation}, used to evaluate the quality of test
data.  To run \sa~to generate test data for the provided \texttt{Inventory} schema, the following command could be used:
\lstinline{java org.schemaanalyst.util.Go -s parsedcasestudy.Inventory generation}.

% Discuss the type of output that the tool can produce

With no other command-line options, \sa~will produce a Java class containing a JUnit test suite with the generated test
data. By default, this class will be created under the \texttt{generatedtest} package and saved in a folder of the same
name in the tool's root directory.  The user may append the \lstinline{--inserts} option to the \lstinline{generation}
command to obtain the generated test data in the form of SQL \texttt{INSERT} statements that are saved in plain text
instead of a JUnit test suite.

% TODO SEKE AND TOSEM data --- GMK QUESTION: Does this still need to be done? (seems like this data is in the table)

% GMK NOTE: In all of the other locations in the paper, a schema name is in texttt, so I did the same in this table

% Calculations to support the final Total row in the right-hand column of this table:

% Number of Tables:
% 2 + 5 + 2 + 28 + 23 + 2 + 2 + 5 + 7 + 8 + 1 + 2 + 2 + 3 + 1 + 1 + 6 + 42 + 6 + 6 + 1 + 2 + 2 + 2 + 2 + 1 + 2 + 1 + 3 + 13 + 4 + 2 + 8 + 10 + 8
% 215

% Number of Columns:

% 3 + 7 + 9 + 129 + 69 + 13 + 10 + 20 + 32 + 52 + 7 + 21 + 13 + 14 + 4 + 3 + 40 + 309 + 49 + 51 + 8 + 7 + 32 + 6 + 9 + 3 + 7 + 5 + 9 + 56 + 43 + 6 + 32 + 67 + 29
% 1174

% Number of Constraints:

% 3 + 7 + 8 + 186 + 29 + 10 + 0 + 19 + 42 + 36 + 4 + 9 + 10 + 23 + 2 + 3 + 5 + 134 + 50 + 5 + 1 + 2 + 2 + 2 + 13 + 3 + 7 + 7 + 14 + 36 + 5 + 8 + 23 + 30 + 31
% 769

% Number of Unique Papers:

% 6

% NOTE: There are commands in the preamble/commands.tex file for all of these numbers --- do not use them directly!

\begin{table*}[t]
  \scriptsize
  \centering
  \vspace*{-.2in}
  \caption{Relational Database Schemas used to Experimentally Evaluate the \textit{SchemaAnalyst} Tool}~\label{tab:schemas}
  \begin{tabular}{llllllllll}
    Schema&Tables&Columns&Constraints&Used In&Schema&Tables&Columns&Constraints&Used In \\
    {\tt ArtistSimilarity}&2&3&3&\cite{mcminn2015effectiveness},\cite{wright2014impact} &
    {\tt JWhoisServer}&6&49&50&\cite{kapfhammer2013search},\cite{mcminn2015effectiveness},\cite{kinneer2015automatically},\cite{wright2013efficient},\cite{wright2014impact},\cite{mcminn2016virtual}\\
    {\tt ArtistTerm}&5&7&7&\cite{mcminn2015effectiveness},\cite{wright2014impact} &
    {\tt MozillaExtensions}&6&51&5&\cite{mcminn2015effectiveness} \\
    {\tt BankAccount} &2&9&8&\cite{kapfhammer2013search},\cite{mcminn2015effectiveness},\cite{wright2014impact} &
    {\tt MozillaPermissions} &1&8&1&\cite{mcminn2015effectiveness},\cite{mcminn2016virtual} \\
    {\tt BioSQL} &28&129&186&\cite{kinneer2015automatically} &
    {\tt NistDML181} &2&7&2&\cite{kapfhammer2013search},\cite{mcminn2015effectiveness} \\
    {\tt BookTown} &23&69&29&\cite{kapfhammer2013search},\cite{mcminn2015effectiveness},\cite{wright2014impact} &
    {\tt NistDML182} &2&32&2&\cite{kapfhammer2013search},\cite{mcminn2015effectiveness},\cite{wright2013efficient}\\
    {\tt BrowserCookies} &2&13&10&\cite{mcminn2015effectiveness} &
    {\tt NistDML183} &2&6&2&\cite{kapfhammer2013search},\cite{mcminn2015effectiveness},\cite{wright2013efficient},\cite{wright2014impact} \\
    {\tt Cloc} &2&10&0&\cite{kapfhammer2013search},\cite{mcminn2015effectiveness},\cite{kinneer2015automatically},\cite{wright2013efficient},\cite{wright2014impact}&
    {\tt NistWeather} &2&9&13&\cite{kapfhammer2013search},\cite{mcminn2015effectiveness},\cite{kinneer2015automatically},\cite{mcminn2016virtual}\\
    {\tt CoffeeOrders} &5&20&19&\cite{kapfhammer2013search},\cite{mcminn2015effectiveness},\cite{wright2014impact},\cite{mcminn2016virtual}&
    {\tt NistXTS748} &1&3&3&\cite{kapfhammer2013search},\cite{mcminn2015effectiveness},\cite{kinneer2015automatically} \\
    {\tt CustomerOrder} &7&32&42&\cite{kapfhammer2013search},\cite{mcminn2015effectiveness}&
    {\tt NistXTS749}
    &2&7&7&\cite{kapfhammer2013search},\cite{mcminn2015effectiveness},\cite{kinneer2015automatically},\cite{wright2014impact}\\
    {\tt DellStore} &8&52&36&\cite{kapfhammer2013search},\cite{mcminn2015effectiveness}&
    {\tt Person} &1&5&7&\cite{kapfhammer2013search},\cite{mcminn2015effectiveness},\cite{mcminn2016virtual}\\
    {\tt Employee} &1&7&4&\cite{kapfhammer2013search},\cite{mcminn2015effectiveness},\cite{mcminn2016virtual}&
    {\tt Products} &3&9&14&\cite{kapfhammer2013search},\cite{mcminn2015effectiveness},\cite{mcminn2016virtual}\\
    {\tt Examination} &2&21&9&\cite{kapfhammer2013search},\cite{mcminn2015effectiveness}&
    {\tt RiskIt}
    &13&56&36&\cite{kapfhammer2013search},\cite{mcminn2015effectiveness},\cite{kinneer2015automatically},\cite{wright2013efficient},\cite{wright2014impact} \\
    {\tt Flights} &2&13&10&\cite{kapfhammer2013search},\cite{mcminn2015effectiveness},\cite{wright2014impact}&
    {\tt StackOverflow} &4&43&5&\cite{mcminn2015effectiveness},\cite{wright2014impact} \\
    {\tt FrenchTowns} &3&14&23&\cite{kapfhammer2013search},\cite{mcminn2015effectiveness}&
    {\tt StudentResidence} &2&6&8&\cite{kapfhammer2013search},\cite{mcminn2015effectiveness} \\
    {\tt Inventory} &1&4&2&\cite{kapfhammer2013search},\cite{mcminn2015effectiveness},\cite{mcminn2016virtual}&
    {\tt UnixUsage} &8&32&23&\cite{kapfhammer2013search},\cite{mcminn2015effectiveness},\cite{kinneer2015automatically},\cite{wright2013efficient},\cite{wright2014impact} \\
    {\tt Iso3166} &1&3&3&\cite{kapfhammer2013search},\cite{mcminn2015effectiveness},\cite{mcminn2016virtual}&
    {\tt Usda} &10&67&30&\cite{kapfhammer2013search},\cite{mcminn2015effectiveness}  \\
    {\tt IsoFlav} &6&40&5&\cite{wright2014impact}&
    {\tt WordNet} &8&29&31&\cite{wright2014impact} \\
    {\tt iTrust} &42&309&134&\cite{mcminn2015effectiveness},\cite{kinneer2015automatically}&
    Total&\numtables&\numcolumns&\numconstraints&\numuniquepapers~(Unique)
  \end{tabular}
  \vspace*{-.1in}
\end{table*}


% Quickly explain the output from performing mutation analysis

If the \lstinline{mutation} command is given to perform mutation analysis on the test data generated by \sa, a folder
called \lstinline{results} will be created in the project's root directory. This folder will contain a comma-separated
value file recording the parameters used in the analysis as well as the mutation score and some additional runtime
information.

% Point the reader to the GitHub site!

The \sa~GitHub page also provides comprehensive documentation, including detailed installation and usage
instructions~\cite{tool}. In addition to featuring a thorough JUnit test suite, the source code of \sa~contains
documentation that aims to aid developers who want extend the tool or better understand certain implementation
decisions.

% NOTE that this content has been moved to an earlier section of the document

% The help menu provided by the \lstinline{--help} option (Figure~\ref{fig:usage}) shows a complete list of \sa's command
% line arguments with descriptions. Table~\ref{tab:args} shows the coverage criteria and data generators provided with
% \sa.


\section{Related Work}\label{sec:relatedwork}
\textit{SchemaAnalyst} has been used in experimental studies, facilitating research into
search-based software testing and mutation testing.  \textit{SchemaAnalyst} has been used to enable
empirical studies appearing in five published works. 
\subsection{Survey of Results}

Kapfhammer et al. compared \textit{SchemaAnalyst} to \textit{DBMonster}
in an experiment using mutation testing over three DBMSs and 25 database 
schemas~\cite{kapfhammer2013search}. The results showed that \textit{SchemaAnalyst} outperformed
\textit{DBMonster} in terms of mutation score and constraint coverage, while remaining competitive
in execution time.

McMinn et al. organized the nine coverage criteria used in \textit{SchemaAnalyst} into an subsumption
hierarchy, and investigated the effectiveness of the criteria in a study across
three DBMSs and 32 database schemas~\cite{mcminn2015effectiveness}.  The results showed mutation
scores as low as $12\%$ for the least stringent criteria, to as high as $96\%$ for the most stringent.

Kinneer et al. conducted a study on the scalability of \textit{SchemaAnalyst}, finding that the tool
scaled well for all realistically sized schemas~\cite{kinneer2015automatically}.

Wright et al. introduced techniques for improving the efficiency of mutation testing for database
schemas using parallelisation~\cite{wright2013efficient}. These techniques were evaluated using two
DBMSs and six database schemas, with one technique resulting in one to ten times performance
improvement for both DBMSs.

McMinn et al. \dots~\cite{mcminn2016virtual}.

Wright et al. used \textit{SchemaAnalyst} to evaluate the influence of ineffective mutants on
database schema mutation analysis in a study that used sixteen database
schemas~\cite{wright2014impact}. Removing the ineffective mutants resulted in efficiency
improvements of up to $33.7\%$.

Chris wrote a thesis about this. Wow, that's a long read~\cite{wright2015mutation}.

\subsection{Related work}
\textit{SchemaAnalyst} uses Korel's Alternating Variable Method (AVM)~\cite{Korel:AVM},
a search technique for optimizing parameter values. AVM find the best settings using 
exploratory and pattern moves. AVM operates on one parameter at a time. The parameter is
altered in different directions in the search space during ht exploratory move phase.  Then,
whichever direction resulted in the most improvement in fitness is taken until the fitness stops 
improving. AVM repeats this process for each parameter under consideration until the fitness is
high enough, or until the search budget is exhausted.

Gordon Fraser and Andrea Arcuri presented \textit{EvoSuite}, a tool for test data generation for
Java programs~\cite{Fraser2011evosuite}.
Like \textit{SchemaAnalyst}, \textit{EvoSuite} uses a search-based approach 
to generate test data. However, \textit{EvoSuite} generates test data for Java programs, where
\textit{SchemaAnalyst} focuses on genreating test data for relational database schemas.

More information on software testing, including coverage criteria and logic testing
can be found in Ammann Offutt's book,``Software Testing''~\cite{ammann2008}.

Silberschatz et al. provide a resource with more information on relational databases,
database schemas, and integrity constraints~\cite{databasebook}.

Phil McMinn provides on overview of search-based software testing~\cite{STVR:STVR294}.

\section{Conclusions and Future Work}\label{sec:conclusion}

Many data-centric services rely on the quality of the underlying data. Much of this data is managed by relational
databases, and the database schema protects the integrity of the data.  Testing the schema for correctness is vital to
ensuring data quality. \sa~is a tool for generating test data for relational database schemas to ensure that the schema
is functioning as intended. Using a search-based technique, \sa~is able to provide high quality test data across
multiple database management systems. \sa~also includes an evaluation framework that provides
\numprovidedschemas~case-study schemas and support for efficient mutation analysis. In addition to being used in
\numuniquepapers~published studies, the presented tool is now available to the public on GitHub~\cite{tool}. With an
open-source license and modular design, \sa~is an extensible platform for search-based test data generation and mutation
testing.


% An example of a floating figure using the graphicx package.
% Note that \label must occur AFTER (or within) \caption.
% For figures, \caption should occur after the \includegraphics.
% Note that IEEEtran v1.7 and later has special internal code that
% is designed to preserve the operation of \label within \caption
% even when the captionsoff option is in effect. However, because
% of issues like this, it may be the safest practice to put all your
% \label just after \caption rather than within \caption{}.
%
% Reminder: the "draftcls" or "draftclsnofoot", not "draft", class
% option should be used if it is desired that the figures are to be
% displayed while in draft mode.
%
%\begin{figure}[!t]
%\centering
%\includegraphics[width=2.5in]{myfigure}
% where an .eps filename suffix will be assumed under latex, 
% and a .pdf suffix will be assumed for pdflatex; or what has been declared
% via \DeclareGraphicsExtensions.
%\caption{Simulation results for the network.}
%\label{fig_sim}
%\end{figure}

% Note that the IEEE typically puts floats only at the top, even when this
% results in a large percentage of a column being occupied by floats.


% An example of a double column floating figure using two subfigures.
% (The subfig.sty package must be loaded for this to work.)
% The subfigure \label commands are set within each subfloat command,
% and the \label for the overall figure must come after \caption.
% \hfil is used as a separator to get equal spacing.
% Watch out that the combined width of all the subfigures on a 
% line do not exceed the text width or a line break will occur.
%
%\begin{figure*}[!t]
%\centering
%\subfloat[Case I]{\includegraphics[width=2.5in]{box}%
%\label{fig_first_case}}
%\hfil
%\subfloat[Case II]{\includegraphics[width=2.5in]{box}%
%\label{fig_second_case}}
%\caption{Simulation results for the network.}
%\label{fig_sim}
%\end{figure*}
%
% Note that often IEEE papers with subfigures do not employ subfigure
% captions (using the optional argument to \subfloat[]), but instead will
% reference/describe all of them (a), (b), etc., within the main caption.
% Be aware that for subfig.sty to generate the (a), (b), etc., subfigure
% labels, the optional argument to \subfloat must be present. If a
% subcaption is not desired, just leave its contents blank,
% e.g., \subfloat[].


% An example of a floating table. Note that, for IEEE style tables, the
% \caption command should come BEFORE the table and, given that table
% captions serve much like titles, are usually capitalized except for words
% such as a, an, and, as, at, but, by, for, in, nor, of, on, or, the, to
% and up, which are usually not capitalized unless they are the first or
% last word of the caption. Table text will default to \footnotesize as
% the IEEE normally uses this smaller font for tables.
% The \label must come after \caption as always.
%
%\begin{table}[!t]
%% increase table row spacing, adjust to taste
%\renewcommand{\arraystretch}{1.3}
% if using array.sty, it might be a good idea to tweak the value of
% \extrarowheight as needed to properly center the text within the cells
%\caption{An Example of a Table}
%\label{table_example}
%\centering
%% Some packages, such as MDW tools, offer better commands for making tables
%% than the plain LaTeX2e tabular which is used here.
%\begin{tabular}{|c||c|}
%\hline
%One & Two\\
%\hline
%Three & Four\\
%\hline
%\end{tabular}
%\end{table}


% Note that the IEEE does not put floats in the very first column
% - or typically anywhere on the first page for that matter. Also,
% in-text middle ("here") positioning is typically not used, but it
% is allowed and encouraged for Computer Society conferences (but
% not Computer Society journals). Most IEEE journals/conferences use
% top floats exclusively. 
% Note that, LaTeX2e, unlike IEEE journals/conferences, places
% footnotes above bottom floats. This can be corrected via the
% \fnbelowfloat command of the stfloats package.




% use section* for acknowledgment
%\section*{Acknowledgment}




% trigger a \newpage just before the given reference
% number - used to balance the columns on the last page
% adjust value as needed - may need to be readjusted if
% the document is modified later
%\IEEEtriggeratref{8}
% The "triggered" command can be changed if desired:
%\IEEEtriggercmd{\enlargethispage{-5in}}

% references section

\nocite{*}

% can use a bibliography generated by BibTeX as a .bbl file
% BibTeX documentation can be easily obtained at:
% http://mirror.ctan.org/biblio/bibtex/contrib/doc/
% The IEEEtran BibTeX style support page is at:
% http://www.michaelshell.org/tex/ieeetran/bibtex/
\bibliographystyle{IEEEtran}
% argument is your BibTeX string definitions and bibliography database(s)
\bibliography{IEEEabrv,bib/bibtex}


% that's all folks
\end{document}


