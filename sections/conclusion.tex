%!TEX root=../icsme2016_tool_paper.tex

% push the rest of the text down to align with the end of the page
\vspace{.5em}
\section{Conclusions and Future Work}\label{sec:conclusion}
\vspace{.5em}

% Review the contributions of the presented tool

Many database-centric services rely on the quality of the underlying data. Much of this data is managed by relational
databases, with the database schema protecting the integrity of the data. Testing the schema for correctness is vital to
ensuring data quality. \sa~is a tool that generates test data for a relational database schema, thereby increasing
confidence in the schema's correctness. Using a search-based technique, \sa~automatically creates high-quality test data
across multiple DBMSs. The presented tool also includes an evaluation framework that provides
\numprovidedschemas~case-study schemas and support for efficient mutation analysis. In addition to being used in
\numuniquepapers~published studies, the presented tool is now available from \sawebsite~\cite{tool}. With an open-source
license and a modular design, \sa~is an extensible tool for search-based test data generation and mutation testing,
enabling the work of both researchers and practitioners.

% Discuss several areas for future work and cast a vision for the future of this tool

In future work, we will evaluate how \sa~helps the people who design and test database schemas.  We plan to
incorporate techniques that generate more readable and realistic data
values~\cite{Afshan2013,McMinn2012,Shahbaz2012,Shahbaz2015}, thus helping humans understand test cases more
easily~\cite{Fraser2015, Fraser2013}. We will also integrate the tool with others that support software maintenance
activities like regression testing~\cite{Kapfhammer2008} and fault localization~\cite{Clark2011}. Next, we will extend
the tool so that it enables the testing of recently developed NoSQL systems. Ultimately, the current version of \sa, our
planned extensions, and the features and studies contributed by the new researchers and industrialists using this
now-released tool will yield a comprehensive approach to testing database-centric applications.
