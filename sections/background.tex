%!TEX root=../icsme2016_tool_paper.tex
\section{Background}\label{sec:background}

% Explain the idea of software testing

Software testing, the process of running a software system to ensure that it functions as intended, is a key part of the
software development lifecycle~\cite{Kapfhammer2004}. If software does not meet users' expectations, then it contains a
fault. Developers can check for these faults by running tests that give the program inputs and check for expected
outputs~\cite{Kapfhammer2010}. If the software produces the expected output for the provided input, then this suggests
that it is functioning correctly. Yet, if it does not perform as anticipated, then the tests may have found \mbox{a
fault}.

% Introduce the idea of a test case and then talk about search-based test data generators

A collection of test cases is called a test suite. A test suite's effectiveness at finding faults is known as its
adequacy, which is assessed by a test suite adequacy criterion.  Writing high-quality tests requires developers to
painstakingly consider the range of possible inputs --- as anticipated, this is a challenging and time-consuming
process~\cite{Fraser2015}. Test data generation reduces the burden on a human tester by (semi-)automatically creating
test inputs. As presented in this paper, search-based test data generation with \sa~employs a fitness function to direct
the tool towards creating high-quality test data~\cite{STVR:STVR294}.

% The fitness function evaluates the quality of the test data, allowing the data generator to iteratively pursue higher
% quality inputs.

% Introduce the idea of mutation testing to assess the quality of a test suite

Mutation adequacy is a criterion that measures the effectiveness of a test suite by modifying the artefact under test to
produce a ``mutant''~\cite{Just2011a}. This change to the entity is meant to simulate a potential fault, so that the
mutant should result in behavior different from that of the original. In this process, the result from running the tests
against the original and mutant artefacts are compared. If the results are the same, then the test suite failed to
detect the seeded fault. Yet, if they are different, then the test suite found the simulated fault, at which point the
mutant is said to be ``killed''. The mutation score is the number of mutants killed divided by the total number of
mutants~\cite{Just2012b}.

%!TEX root=../icsme2016_tool_paper.tex
\begin{figure}[t]
\centering
\scalebox{0.8}{
\begin{tabular}{r|l|}
\hhline{~-}
1 & \texttt{CREATE TABLE Inventory}\\
2 & \texttt{(}\\
3 & \texttt{  id INT PRIMARY KEY,}\\
4 & \texttt{  product VARCHAR(50) UNIQUE,}\\
5 & \texttt{  quantity INT,}\\
6 & \texttt{  price DECIMAL(18,2)}\\
7 & \texttt{);}\\
\hhline{~-}
\end{tabular}
}
\caption{\label{fig:schema}The Inventory relational database schema}
\vspace*{-1em}
\end{figure}


% Give all of the relevant background about relational databases

% GMK NOTE: I cut this sentence to shorten this content and save space
% Relational databases are databases whose data entries can refer to one another.

Managed by applications called database management systems (DBMSs), a relational database is a collection of connected
data~\cite{databasebook}. The database schema is the artefact that lays out the structure of the database, organising it
into tables and columns.  The schema can also define integrity constraints, or rules that candidate data must meet
before the DBMS will accept it. If the pending data violates an integrity constraint specified by the schema, then the
DBMS rejects it as invalid.  Figure~\ref{fig:schema} furnishes a database schema for recording the number of products
kept in an inventory. This schema defines one table, called \texttt{Inventory}, with four columns.  The \texttt{id}
column on line $3$ is annotated with the \texttt{PRIMARY KEY} constraint, indicating that data inserted into it cannot
be left missing or unknown, and that the values in this column must be unique. If the \texttt{PRIMARY KEY} was left out
of the database schema, then multiple items could be entered with the same \texttt{id} value, potentially resulting in
the incorrect application behavior.

% this mistake could result in the wrong item being shipped to the customer in an industrial application.
