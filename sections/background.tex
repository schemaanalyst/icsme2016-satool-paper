\section{Background}\label{sec:background}
Software testing is a vital part of the software development lifecycle, the
process of ensuring that the software functions as intended.  When this is not the case, then a fault
is said to exist in. Developers can check for faults by using test cases~\cite{ammann2008}, pairs of inputs 
expected outputs. If the software produces the expected output for the provided input, then this is evidence
that the software is functioning as expected.  If the software does not perform as expected however, then a
fault has been discovered.

A collection of test cases is called a test suite. A test suite's effectiveness at finding faults is
called test suite adequacy, which is measured by a test suite adequacy criterion.
Writing high quality test suites requires developers to painstakingly consider the
range of possible inputs, and is a time consuming process. Test data generation is used to
help by generating test case inputs automatically, reducing the burden on a human expert. A test data
generator is an artefact that generates test data automatically.
Search-based test data generators are test data generators that use a fitness function to guide the
test data generator towards producing high-quality test data. The fitness function evaluates the
quality of the test data, allowing the test data generator to iteratively pursue higher quality test 
data.

Mutation testing is a test suite adequacy criterion that measures the effectiveness of a test suite. In
mutation testing, the artefact under test is modified to produce a ``mutant''. The
change is meant to simulate an artificial fault, and the mutant artifact is expected to result in
different behavior from the original. The result of the test suite from the original and mutant
artefacts are compared. If the results are the same, then the test suite failed to detect the
artificially seeded fault. If the results are different however, then the test suite was able to
differentiate the two artifacts, finding the simulated fault, and the mutant is said to be ``killed''.
The mutation score is $\frac{\mathit{MutantsKilled}}{\mathit{MutantsTotal}}$.

%!TEX root=../icsme2016_tool_paper.tex
\begin{figure}[t]
\centering
\scalebox{0.8}{
\begin{tabular}{r|l|}
\hhline{~-}
1 & \texttt{CREATE TABLE Inventory}\\
2 & \texttt{(}\\
3 & \texttt{  id INT PRIMARY KEY,}\\
4 & \texttt{  product VARCHAR(50) UNIQUE,}\\
5 & \texttt{  quantity INT,}\\
6 & \texttt{  price DECIMAL(18,2)}\\
7 & \texttt{);}\\
\hhline{~-}
\end{tabular}
}
\caption{\label{fig:schema}The Inventory relational database schema}
\vspace*{-1em}
\end{figure}


A database is a software system for storing and maintaining collections of data. Databases
are managed by applications called database management systems (DMBS)s. Relational databases are
databases whose data entries can refer to one another. The database schema is the artefact that
lays out the structure of the database, which is organized into tables and columns. The schema can also
define integrity constraints, which are rules that data submitted to the database must meet to be
accepted. If the data violates an integrity constraint specified by the schema, then it is rejected as
invalid. Figure~\ref{fig:schema} shows a simple database schema for recording the number of products
kept in an inventory. This schema defines one table, called \texttt{Inventory}, with four columns.
The \texttt{id} column on line three is given the \texttt{PRIMARY KEY} constraint, which means that
data inserted into this column cannot be left missing or unknown, and that the values in this column
must be unique.
