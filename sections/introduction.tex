\section{Introduction}\label{sec:intro}

% Introduction to the importance of databases

Healthcare, science, and commerce often rely on information that is stored in
databases~\cite{kapfhammer2007comprehensive}.  When this data is incorrect, passengers can have their flights delayed or
patients may receive the wrong medication~\cite{databasebook}.  In addition to documenting the structure of and
connections between data entries, relational databases furnish a means for protecting the correctness of the data that
they store.  In particular, the relational database schema is the artifact that is responsible for safeguarding the
integrity of a relational database. The vital role of the database schema makes the testing of it a task of vital
importance.

% Comments on NoSQL systems and then explain why the relational DBMS is still important

While non-relational ``NoSQL'' database systems have been gaining in popularity, relational databases remain pervasive.
For instance, Skype, the widely used video-call software, uses the PostgreSQL database management system
(DBMS)~\cite{postgres} while Google makes use of the SQLite DBMS in Android phones~\cite{sqlite}.  Additionally,
according to DB-Engines.com, the top three most popular DBMSs are relational in nature~\cite{dbrank}; also, the 968,277
questions asked on StackExchange about relational databases show the demand for their support~\cite{stackexchange}.

% Explain more about the tool and cite some of its benefits

% GMK NOTE: The phrase "enables verifying" is awkward, but difficult to rephrase

\sa~is a tool for generating high-quality test data in support of database schema testing. Using a search-based approach
that evaluates fitness to incrementally improve test data~\cite{Korel:AVM}, {\sa} discovers test data instances that
comprehensively exercise the database schema.  {\sa} includes an evaluation framework with a collection of real-world
case study schemas, as well as a mutation analysis system that enables verifying the quality of the generated test data
based on its capability to detect systematically seeded faults.  Additionally, {\sa} is extensible, well documented, and
available on GitHub under an open-source license~\cite{tool}.

% Summarize when the tool has been used and comment on the benefits of using it

{\sa} has been used to support research studies focusing on both search-based software
testing~\cite{kapfhammer2013search,mcminn2015effectiveness,kinneer2015automatically} and mutation
testing~\cite{wright2013efficient,wright2014impact,wright2015mutation,mcminn2016virtual}.  In addition to describing the
implementation of \sa~and overviewing its efficiency and effectiveness, this paper inaugurates the public release of
this testing tool. Since past studies have shown the benefits of using the presented open-source tool instead of
competing systems, this paper argues that \sa~is ready to enhance practitioners' testing of database schemas.  In
summary, the key contributions of this paper are as follows:

% Key Contributions:

\begin{itemize}

  % The SchemaAnalyst tool itself

  \item {\sa}, an extensible, efficient, and effective tool that generates test data for database schemas
    (Section~\ref{sec:technique}).

  % The surrounding framework that includes the schemas needed for empirical study and the mutation analysis

  \item In support of researchers, a comprehensive evaluation framework, including relational schemas suitable for
    further empirical study and mutation analysis tools supporting the assessment of test data quality
    (Section~\ref{sec:technique}).

  % Documentation of the key features of the tool (e.g., the command lines)

  \item Aiding both researchers and practitioners, documentation explaining the features and usage of the tool
    (Section~\ref{sec:implementation}).

  % Due to space constraints, a brief overview of past results obtained when using SchemaAnalyst

  \item Confirming the scalability and applicability of \sa, a survey of prior empirical results
    (Section~\ref{sec:relatedwork}).

\end{itemize}
