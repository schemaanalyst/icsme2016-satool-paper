\section{Introduction}\label{sec:intro}

Healthcare, science, and commerce often rely on information that is stored in
databases~\cite{kapfhammer2007comprehensive}.  When this data is incorrect, passengers can have their flights delayed or
patients may receive the wrong medication~\cite{databasebook}.  In addition to documenting the structure of and
connections between data entries, relational databases furnish a means for protecting the correctness of the data that
they store.  In particular, the relational database schema is the artifact that is responsible for safeguarding the
integrity of a relational database. The vital role of the database schema makes testing the schema a task of vital
importance.

While non-relational ``NoSQL'' database systems have been gaining in popularity, relational databases remain pervasive.
For instance, Skype, the widely used video-call software, uses the PostgreSQL database management system
(DBMS)~\cite{postgres} while Google makes use of the SQLite DBMS in Android phones~\cite{sqlite}.  Additionally,
according to DB-Engines.com, the top three most popular DBMSs are relational in nature~\cite{dbrank}; also, the 968,277
questions asked on StackExchange about relational databases show the demand for their support~\cite{stackexchange}.

\sa~is a tool for generating high-quality test data in support of database schema testing. Using a search-based approach
that evaluates fitness to incrementally improve test data~\cite{Korel:AVM}, {\sa} discovers test data instances that
comprehensively exercise the database schema.  {\sa} includes an evaluation framework with a collection of real-world
case study schemas, as well as a mutation analysis feature that enables verifying the quality of the generated test data
based on its ability to detect artificially seeded faults.  Additionally, {\sa} is extensible, well documented, and
available on GitHub under an open-source license~\cite{tool}.

{\sa} has been used to support research studies involving search-based software
testing~\cite{kapfhammer2013search,mcminn2015effectiveness,kinneer2015automatically}, and mutation
testing~\cite{wright2013efficient,wright2014impact,wright2015mutation,mcminn2016virtual}.

In summary, the key contributions of this paper are as follows:

%Key Contributions:
\begin{itemize}
\item {\sa}, an extensible tool for test data generation Section~\ref{sec:technique}.
\item A comprehensive evaluation framework inc. schemas and mutation analysis tools Section~\ref{sec:technique}.
\item Documentation explaining the features and usage of the tool Section~\ref{sec:implementation}.
\item A survey of results obtained with {\sa} Section~\ref{sec:relatedwork}.
\end{itemize}
