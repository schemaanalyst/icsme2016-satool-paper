\begin{figure}
\lstinputlisting{figures/Usage.txt}
\caption{\label{fig:usage} The first section of the \sa~help menu.}
\end{figure}

\section{Implementation}\label{sec:implementation}
\subsection{Design}

% Introduce the design and implementation of the SchemaAnalyst tool; closing with a commentary about the fact that the
% chosen SQL parser is a commercial product that we cannot release on GitHub.

\sa~is implemented in the Java programming language.  Designed with extensibility in mind, the \sa~tool is divided into
$13$ packages, which this paper briefly overviews. The \texttt{sqlrepresentation} package provides an intermediate Java
representation of data structures in relational databases, fully modelling database tables, columns, expressions, data
types, integrity constraints, and other relevant entities. These objects enable \sa~to support multiple DBMSs (i.e.,
\sqlite, \postgres, and \hypersql), and, additionally, allow for the inclusion of new DBMSs. The tool also contains the
\texttt{sqlparser} package that wraps the General SQL Parser~\cite{generalsqlparser}, thus enabling the effective
conversion of a schema expressed in the Structured Query Language (SQL) to the tool's intermediate representation. As
this SQL parser is a commercial product, the open-source version of \sa~does not provide it for download on GitHub.
Therefore, users can experiment with \sa~by either testing the provided schemas or (automatically or manually)
converting a new schema to the tool's internal SQL representation.

% Go into additional details about some of the packages (breaking up an otherwise too-long paragraph)

The \texttt{testgeneration} package provides a representation of test suites and test cases, as well as test
requirements and the nine provided coverage criteria.  The \texttt{data} package furnishes the three provided test data
generators, as well as various generic data-type representations for use during test data generation. The \texttt{dbms}
package provides support for the three currently supported DBMSs, and includes the classes that enable interaction with
installed DBMSs. The \texttt{sqlwriter} package provides support for creating SQL statements for use with DBMSs, and is
used with the \texttt{javawriter} package to encode the generated test data as a JUnit test suite.  The
\texttt{mutation} package provides the mutation analysis functionality, including the 14 provided mutation operators,
mutant equivalence and reduction features~\cite{wright2014impact}, and virtual test suite
executors~\cite{mcminn2016virtual}.

\subsection{Usage}

% NOTE: The lstinline command must be on a full line and no line wrapping is allowed (will cause compiler error)

\sa~is publicly available on GitHub under an open-source license~\cite{tool}. After cloning the Git
repository, the project can be built using Gradle with the following command from the project's root directory:
\lstinline{./gradlew compile}. After the project compiles, set the \lstinline{CLASSPATH} variable using the following
command: \lstinline{export CLASSPATH="build/classes/main:build/lib/*:lib/*:."}.

Optionally, the user can install the \postgres, \sqlite, and \hypersql~DBMSs. Since \sqlite~does not require
configuration on the machine running \sa, it is included as the default option. If installed, \sa~can run test data on
the installed databases. \sa~also supports a virtual DBMS executor allowing SQL statements to be simulated. If the use
of the actual DBMS is desired, the user should refer to online documentation on configuring the DBMSs~\cite{tool}.

\begin{table}[t]
\centering
\caption{Key Features Provided by \textit{SchemaAnalyst}.}
\label{tab:args}

\begin{tabular}{rr}
  \begin{minipage}{1.25in}

    \begin{tabular}{l}
    Coverage Criteria \\
    \midrule
      APC                                    \\
      ICC                                    \\
      AICC                                   \\
      CondAICC                               \\
      ClauseAICC                             \\
      UCC                                    \\
      AUCC                                   \\
      NCC                                    \\
      ANCC
    \end{tabular}

  \end{minipage} &

  \begin{minipage}{1.25in}

    \begin{tabular}{l}
    Data Generators \\
    \midrule
      AVM -- Random Restart  \\
      AVM -- Default Restart \\
      AICC                                   \\
      CondAICC                               \\
      ClauseAICC                             \\
      UCC                                    \\
      AUCC                                   \\
      NCC                                    \\
      ANCC
    \end{tabular}

  \end{minipage}
\end{tabular}
\vspace*{-.25in}

\end{table}




Usage instructions for \sa~can be obtained by: \lstinline{java org.schemaanalyst.util.Go --help}.
Figure~\ref{fig:usage} shows a snippet of this menu.  As indicated by the help display, \sa~first
expects options indicating the desired schema, coverage criterion, data generator, and DBMS\@. Defaults are provided for
all of these options except for the schema option, which is required. The user must then give a command.  The two
supported commands are \lstinline{generation}, used to generate test data, and \lstinline{mutation}, used to evaluate
the quality of test data.  To use \sa~to generate test data for the provided Inventory schema, the
following command could be used: \lstinline{java org.schemaanalyst.util.Go -s parsedcasestudy.Inventory generation}.

With no other command line options, \sa~will create a Java class containing a JUnit test suite with
the generated test data. By default, this class will be created under the \texttt{generatedtest} package, and saved in a
folder of the same name in the root directory of the tool.  The user may append the \lstinline{--inserts} option to the
\lstinline{generation} command to obtain the generated test data in the form of SQL \texttt{INSERT} statements in plain
text instead of a JUnit test suite.

% TODO SEKE AND TOSEM data --- GMK QUESTION: Does this still need to be done? (seems like this data is in the table)

% GMK NOTE: In all of the other locations in the paper, a schema name is in texttt, so I did the same in this table

% Calculations to support the final Total row in the right-hand column of this table:

% Number of Tables:
% 2 + 5 + 2 + 28 + 23 + 2 + 2 + 5 + 7 + 8 + 1 + 2 + 2 + 3 + 1 + 1 + 6 + 42 + 6 + 6 + 1 + 2 + 2 + 2 + 2 + 1 + 2 + 1 + 3 + 13 + 4 + 2 + 8 + 10 + 8
% 215

% Number of Columns:

% 3 + 7 + 9 + 129 + 69 + 13 + 10 + 20 + 32 + 52 + 7 + 21 + 13 + 14 + 4 + 3 + 40 + 309 + 49 + 51 + 8 + 7 + 32 + 6 + 9 + 3 + 7 + 5 + 9 + 56 + 43 + 6 + 32 + 67 + 29
% 1174

% Number of Constraints:

% 3 + 7 + 8 + 186 + 29 + 10 + 0 + 19 + 42 + 36 + 4 + 9 + 10 + 23 + 2 + 3 + 5 + 134 + 50 + 5 + 1 + 2 + 2 + 2 + 13 + 3 + 7 + 7 + 14 + 36 + 5 + 8 + 23 + 30 + 31
% 769

% Number of Unique Papers:

% 6

% NOTE: There are commands in the preamble/commands.tex file for all of these numbers --- do not use them directly!

\begin{table*}[t]
  \scriptsize
  \centering
  \vspace*{-.2in}
  \caption{Relational Database Schemas used to Experimentally Evaluate the \textit{SchemaAnalyst} Tool}~\label{tab:schemas}
  \begin{tabular}{llllllllll}
    Schema&Tables&Columns&Constraints&Used In&Schema&Tables&Columns&Constraints&Used In \\
    {\tt ArtistSimilarity}&2&3&3&\cite{mcminn2015effectiveness},\cite{wright2014impact} &
    {\tt JWhoisServer}&6&49&50&\cite{kapfhammer2013search},\cite{mcminn2015effectiveness},\cite{kinneer2015automatically},\cite{wright2013efficient},\cite{wright2014impact},\cite{mcminn2016virtual}\\
    {\tt ArtistTerm}&5&7&7&\cite{mcminn2015effectiveness},\cite{wright2014impact} &
    {\tt MozillaExtensions}&6&51&5&\cite{mcminn2015effectiveness} \\
    {\tt BankAccount} &2&9&8&\cite{kapfhammer2013search},\cite{mcminn2015effectiveness},\cite{wright2014impact} &
    {\tt MozillaPermissions} &1&8&1&\cite{mcminn2015effectiveness},\cite{mcminn2016virtual} \\
    {\tt BioSQL} &28&129&186&\cite{kinneer2015automatically} &
    {\tt NistDML181} &2&7&2&\cite{kapfhammer2013search},\cite{mcminn2015effectiveness} \\
    {\tt BookTown} &23&69&29&\cite{kapfhammer2013search},\cite{mcminn2015effectiveness},\cite{wright2014impact} &
    {\tt NistDML182} &2&32&2&\cite{kapfhammer2013search},\cite{mcminn2015effectiveness},\cite{wright2013efficient}\\
    {\tt BrowserCookies} &2&13&10&\cite{mcminn2015effectiveness} &
    {\tt NistDML183} &2&6&2&\cite{kapfhammer2013search},\cite{mcminn2015effectiveness},\cite{wright2013efficient},\cite{wright2014impact} \\
    {\tt Cloc} &2&10&0&\cite{kapfhammer2013search},\cite{mcminn2015effectiveness},\cite{kinneer2015automatically},\cite{wright2013efficient},\cite{wright2014impact}&
    {\tt NistWeather} &2&9&13&\cite{kapfhammer2013search},\cite{mcminn2015effectiveness},\cite{kinneer2015automatically},\cite{mcminn2016virtual}\\
    {\tt CoffeeOrders} &5&20&19&\cite{kapfhammer2013search},\cite{mcminn2015effectiveness},\cite{wright2014impact},\cite{mcminn2016virtual}&
    {\tt NistXTS748} &1&3&3&\cite{kapfhammer2013search},\cite{mcminn2015effectiveness},\cite{kinneer2015automatically} \\
    {\tt CustomerOrder} &7&32&42&\cite{kapfhammer2013search},\cite{mcminn2015effectiveness}&
    {\tt NistXTS749}
    &2&7&7&\cite{kapfhammer2013search},\cite{mcminn2015effectiveness},\cite{kinneer2015automatically},\cite{wright2014impact}\\
    {\tt DellStore} &8&52&36&\cite{kapfhammer2013search},\cite{mcminn2015effectiveness}&
    {\tt Person} &1&5&7&\cite{kapfhammer2013search},\cite{mcminn2015effectiveness},\cite{mcminn2016virtual}\\
    {\tt Employee} &1&7&4&\cite{kapfhammer2013search},\cite{mcminn2015effectiveness},\cite{mcminn2016virtual}&
    {\tt Products} &3&9&14&\cite{kapfhammer2013search},\cite{mcminn2015effectiveness},\cite{mcminn2016virtual}\\
    {\tt Examination} &2&21&9&\cite{kapfhammer2013search},\cite{mcminn2015effectiveness}&
    {\tt RiskIt}
    &13&56&36&\cite{kapfhammer2013search},\cite{mcminn2015effectiveness},\cite{kinneer2015automatically},\cite{wright2013efficient},\cite{wright2014impact} \\
    {\tt Flights} &2&13&10&\cite{kapfhammer2013search},\cite{mcminn2015effectiveness},\cite{wright2014impact}&
    {\tt StackOverflow} &4&43&5&\cite{mcminn2015effectiveness},\cite{wright2014impact} \\
    {\tt FrenchTowns} &3&14&23&\cite{kapfhammer2013search},\cite{mcminn2015effectiveness}&
    {\tt StudentResidence} &2&6&8&\cite{kapfhammer2013search},\cite{mcminn2015effectiveness} \\
    {\tt Inventory} &1&4&2&\cite{kapfhammer2013search},\cite{mcminn2015effectiveness},\cite{mcminn2016virtual}&
    {\tt UnixUsage} &8&32&23&\cite{kapfhammer2013search},\cite{mcminn2015effectiveness},\cite{kinneer2015automatically},\cite{wright2013efficient},\cite{wright2014impact} \\
    {\tt Iso3166} &1&3&3&\cite{kapfhammer2013search},\cite{mcminn2015effectiveness},\cite{mcminn2016virtual}&
    {\tt Usda} &10&67&30&\cite{kapfhammer2013search},\cite{mcminn2015effectiveness}  \\
    {\tt IsoFlav} &6&40&5&\cite{wright2014impact}&
    {\tt WordNet} &8&29&31&\cite{wright2014impact} \\
    {\tt iTrust} &42&309&134&\cite{mcminn2015effectiveness},\cite{kinneer2015automatically}&
    Total&\numtables&\numcolumns&\numconstraints&\numuniquepapers~(Unique)
  \end{tabular}
  \vspace*{-.1in}
\end{table*}


If the \lstinline{mutation} command is given to perform mutation analysis on the test data generated by
\sa, a folder called \lstinline{results} will be created in the project's root directory.
This folder will contain a comma-separated list file recording the parameters used in the analysis as well
as the mutation score along with some runtime information.

The help menu provided by the \lstinline{--help} option (Figure~\ref{fig:usage}) shows a complete list of \sa's command
line arguments with descriptions. Table~\ref{tab:args} shows the coverage criteria and data generators provided with
\sa. The \sa~GitHub page also provides comprehensive documentation, including more detailed installation and usage
instructions~\cite{tool}. The source code of \sa~has also been thoroughly documented, and includes a JUnit test suite to
maintain reliability.

