\section{Implementation}
\subsection{Design}
\textit{SchemaAnalyst} is implemented as a Java program.  Designed with extensibility in mind, the
\textit{SchemaAnalyst} tool is divided into 13 packages. The \texttt{sqlrepresentation} package
provides an intermediate Java representation of data structures in relational databases,
providing representations for database tables, columns, expressions, datatypes, integrity
constraints, and other relevant objects. A \texttt{sqlparser} package is provided that utilizes the
General SQL Parser~\cite{} to convert schemas expressed in the Structured Query Language (SQL) to the
Java intermediate representation. The \texttt{testgeneration} package provides a representation of test
suites and test cases, as well as test requirements and the nine provided coverage criteria. The
\texttt{data} package furnishes the three provided test data generators, as well as various generic
datatype representations for use in test data generation. The \texttt{dbms} package provides support
for the three currently supported DBMSs, and includes the classes that enable interaction with
installed DBMSs. The \texttt{sqlwriter} package provides support for creating SQL statements for use
with DBMSs, and is used with the \texttt{javawriter} package to encode the generated test data as a
JUnit test suite. The \texttt{mutation} package provides the mutation analysis functionality, 
including the 14 provided mutation operators, mutant equivalence and reduction features, and virtual
test suite executors.

\subsection{Usage}

\textit{SchemaAnalyst} is publicly available on GitHub under an open-source license~\cite{tool}. After
cloning the Git repository, the project can be built using Gradle with the following command from the
project's root directory: \lstinline{./gradlew compile}. After the project compiles, set the
\lstinline{CLASSPATH} variable using the following command: 
\lstinline{export CLASSPATH="build/classes/main:build/lib/*:lib/*:."}. 

Usage instructions for \textit{SchemaAnalyst} can now be obtained by: 
\lstinline{java org.schemaanalyst.util.Go --help}. Figure~\ref{fig:usage} shows a snippet of this menu.
As indicated by the help display, \textit{SchemaAnalyst} first expects options indicating the desired
schema, coverage criterion, data generator, and DBMS. Defaults are provided for all of these options
except for the schema option, which is required. The user must then give a command.  The two supported
commands are \lstinline{generation}, used to generate test data, and \lstinline{mutation}, used to
evaluate the quality of test data.
To use \textit{SchemaAnalyst} to generate test data for the provided Inventory schema, the following
command could be used: 
\lstinline{java org.schemaanalyst.util.Go -s parsedcasestudy.Inventory generation}.

\begin{figure}
\lstinputlisting{figures/Usage.txt}
\caption{\label{fig:usage} The first section of the \textit{SchemaAnalyst} help screen.}
\end{figure}

With no other command line options, \textit{SchemaAnalyst} will create a Java class containing a JUnit test
suite with the generated test data. By default, this class will be created under the \texttt{generatedtest}
package, and saved in a folder of the same name in the root directory of the tool.

The user may append the \lstinline{--inserts} option to the \lstinline{generation} command to obtain the
generated test data in the form of SQL \texttt{INSERT} statements in plain text instead of a JUnit test suite.
The help menu provided by the \lstinline{--help} option shows a complete list of \textit{SchemaAnalyst}'s
command line arguments with descriptions. The \textit{SchemaAnalyst} GitHub page also provides comprehensive
documentation, including more detailed installation and usage instructions. The source code of
\textit{SchemaAnalyst} has also been thoroughly documented, and 
includes a JUnit test suite to maintain reliability.

\begin{table}[]
\centering
\caption{Coverage criteria and data generators supported by \textit{SchemaAnalyst}.}
\label{tab:args}
\begin{tabular}{l|l}
\multicolumn{1}{c|}{Coverage Criteria} & Data Generator         \\ \hline
APC                                    & avs                    \\
ICC                                    & avsDefaults            \\
AICC                                   & random                 \\
CondAICC                               & randomDefaults         \\
ClauseAICC                             & directedRandom         \\
UCC                                    & directedRandomDefaults \\
AUCC                                   &                        \\
NCC                                    &                        \\
ANCC                                   &                       
\end{tabular}
\end{table}
