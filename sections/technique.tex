\section{\textit{SchemaAnalyst}}

While verifying the accuracy of the database schema is critical for protecting data integrity, manually
creating test data is expensive and time consuming. \textit{SchemaAnalyst} uses a search-based
approach in order to generate test data automatically. After being given an input schema, a coverage
criterion first generates a collection of test requirements. These test requirements are the rules that
the test data must fulfill. A test requirement for the \texttt{Inventory} schema in
Figure~\ref{fig:schema} might be, ``the \texttt{PRIMARY KEY} constraint on line three must be
violated''. A coverage criterion provides a way to systematically generate test requirements. One
example coverage criterion is Integrity Constraint Coverage (ICC). This criterion generates two test
requirements for every integrity constraint in the schema, one requiring that the constraint is
satisfied, and one requiring that the constraint is violated.

\textit{SchemaAnalyst} then generates test data
to satisfy the test requirements using Korel's Alternating Variable Method (AVM)~\cite{Korel:AVM}.

\begin{enumerate}
\item generating requirements
\item generating test data
\item evaluating the test suite
\end{enumerate}

%!TEX root=../icsme2016_tool_paper.tex
\newcommand{\mx}[1]{\mathbf{\bm{#1}}} % Matrix command
\newcommand{\vc}[1]{\mathbf{\bm{#1}}} % Vector command

% Define the layers to draw the diagram
\pgfdeclarelayer{background}
\pgfdeclarelayer{foreground}
\pgfsetlayers{background,main,foreground}

% Define block styles used later

\tikzstyle{sensor}=[draw, fill=black!10, text width=5em,
    text centered, minimum height=2.5em,drop shadow]
\tikzstyle{ann} = [above, text width=5em, text centered]
\tikzstyle{wa} = [sensor, text width=10em, fill=black!30,
    minimum height=6em, rounded corners, drop shadow]
\tikzstyle{sc} = [sensor, text width=13em, fill=red!30,
    minimum height=10em, rounded corners, drop shadow]

% Define distances for bordering
\def\blockdist{1.5}
\def\edgedist{2.5}

\begin{figure}[t]
  \centering

\begin{tikzpicture}[thick,scale=0.85, every node/.style={scale=0.85}]
    \node (wa) [wa]  {\textit{SchemaAnalyst}};
    \path (wa.west)+(-\blockdist,-1.25) node (asr1) [sensor] {Database Schema};

    \path (wa.west)+(-\blockdist,1.25) node (asr2)[sensor] {Coverage Criterion};
    \path (wa.west)+(-\blockdist,0.0) node (dots)[sensor] {Data Generator};

    % \path (dots.west)+(-2.65*\blockdist,-0.0) node (dataa) [sensor] {Data Generator};
    % \path (asr2.west)+(-2.65*\blockdist,-0.0) node (criteriona) [sensor] {Coverage Criterion};

    \path (wa.east)+(\blockdist,0) node (vote) [sensor] {Test Suite};

    \path [draw, ->] (asr1.east) -- node [above] {}
        (wa.200);
    \path [draw, ->] (asr2.east) -- node [above] {}
        (wa.160);
    \path [draw, ->] (dots.east) -- node [above] {}
        (wa.180);
    \path [draw, ->] (wa.east) -- node [above] {}
        (vote.west);

    % \path [draw, ->] (dataa.east) -- node [above] {}
    %     (dots.west);
    % \path [draw, ->] (criteriona.east) -- node [above] {}
    %     (asr2.west);
    % \path [draw, ->] (schemaa.east) -| node [above] {}
    %     (doubler.230);
    %

\end{tikzpicture}

\caption{\label{fig:schemaanalyst}The inputs and outputs of the \textit{SchemaAnalyst} tool\vspace*{-1.75in}}

\end{figure}

