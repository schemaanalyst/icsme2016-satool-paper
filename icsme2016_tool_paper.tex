\documentclass[conference]{IEEEtran}

% Packages that Cody added start here
\usepackage{tikz}
\usetikzlibrary{shapes,arrows,shadows,positioning,calc}
\usepackage{listings}
\usepackage{textcomp}
\usepackage[T1]{fontenc}
\lstset{basicstyle=\small\ttfamily,
columns=flexible,
breaklines=true,
upquote=true
}
\usepackage{hhline}
% Packages that Cody added end here

% Adding in extra packages and commands that help to write about SchemaAnalyst
%!TEX root=../icsme2016_tool_paper.tex

\usepackage{adjustbox}
\usepackage{array}
\usepackage{booktabs}
\usepackage{hhline}
\usepackage{multirow}
\usepackage{ragged2e}
\usepackage{url}
\usepackage{xspace}
\usepackage{textcomp}
\usepackage{balance}

\usepackage[usenames,dvipsnames]{colortbl}

%% allows us to write a bit more ...
\usepackage{times}

\input{preamble/commands}

\begin{document}

\title{\textit{SchemaAnalyst}: Search-based Test Data Generation for Relational Database Schemas}

\author{\IEEEauthorblockN{Phil McMinn and Chris Wright}
\IEEEauthorblockA{University of Sheffield}
\and
\IEEEauthorblockN{Cody Kinneer, Colton McCurdy, Michael Camara\\ and Gregory M. Kapfhammer}
\IEEEauthorblockA{Allegheny College}}
\maketitle

\begin{abstract}

Data stored in relational databases plays a vital role in many aspects of life.  When this data in incorrect, the
services that depend on that data may be compromised.  The database schema is the artefact responsible for maintaining
the integrity of data. Because of this critical function, properly testing the database schema is a task of great
importance.  \sa~is a tool to support testing this vital software component.  \sa~uses a search-based approach to
generate high-quality test data for database schemas. \sa~is extensible, includes a comprehensive evaluation framework
for evaluating the quality of test data, and includes documentation. This paper presents the release of \sa~to enhance
the ability of practitioners to test database schemas.

\end{abstract}

\section{Introduction}

\begin{enumerate}
\item motivation
\item summary of approach
\item overview of the demo
\end{enumerate}

Key Contributions:
\begin{itemize}
\item \textit{SchemaAnalyst}, an extensible tool for test data generation.
\item A comprehensive evaluation framework inc. schemas and mutation analysis tools.
\item Documenation explaining the features and usage of the tool.
\end{itemize}

\section{Background}
Software testing is a vital part of the software development lifecycle.  Software testing is the
process of ensuring that the software functions as intended.  When this is not the case, then a fault
is said to exist in the software. Developers can check for faults by using test cases. A test case is
an input provided to the software, and the correct output that is expected. If the software produces
the expected output for the provided input, then this is evidence that the software is functioning as
expected.  If the software does not perform as expected however, then a fault has been discovered.

A collection of test cases is called a test suite. A test suite's effectiveness at finding faults is
called test suite adequacy, which is measured by a test suite adequacy criterion.
Ideally, developers could provide a test case for every possible input to the software to ensure
that it always works as intended. However, since
software system are often complicated, it is nearly always impossible to test every possible
input to the software. Instead, developers try to balance writing a high adequacy test suite
with the costs of writing additional test cases.

Writing high quality suites requires developers to painstakingly consider the
range of possible inputs, and is a time consuming process. Test data generation is used to
help by generating test case inputs automatically, reducing the burden on a human expert. A test data
generator is an artefact that generates test data automatically.

Search-based test data generators are test data generators that use a fitness function to guide the
test data generator towards producing high quality test data. The fitness function evaluates the
quality of the test data, allowing the test data generator to progressively pursue higher quality test 
data.

Mutation testing is a test suite adequacy criterion that measures the effectiveness of a test suite. In
mutation testing, the artefact under test is randomly modified to produce a ``mutant''. The random
change is meant to simulate an artificial fault, and the mutant artifact is expected to result in
different behavior from the original. The result of the test suite from the original and mutant
artefacts are compared. If the results are the same, then the test suite failed to detect the
artificially seeded fault. If the results are different however, then the test suite was able to
differentiate the two artifacts, finding the simulated fault, and the mutant is said to be ``killed''.

\section{\textit{SchemaAnalyst}}

While verifying the accuracy of the database schema is critical for protecting data integrity, manually
creating test data is expensive and time consuming. \textit{SchemaAnalyst} uses a search-based
approach in order to generate test data automatically. Figure~\ref{fig:schemaanalyst} provides a
high-level overview of the \textit{SchemaAnalyst} system.
After being given an input schema, a coverage criterion first generates a collection of test 
requirements. These test requirements are the rules that
the test data must fulfill. A test requirement for the \texttt{Inventory} schema in
Figure~\ref{fig:schema} might be, ``the \texttt{PRIMARY KEY} constraint on line three must be
violated''. A coverage criterion provides a way to systematically generate test requirements. One
example coverage criterion is Integrity Constraint Coverage (ICC). This criterion generates two test
requirements for every integrity constraint in the schema, one requiring that the constraint is
satisfied, and one requiring that the constraint is violated. \textit{SchemaAnalyst} includes nine
coverage criteria, which are explained in detail in~\cite{mcminn2015effectiveness}.

\textit{SchemaAnalyst} then generates test data
to satisfy the test requirements using a test data generator. The default test data generator used by
\textit{SchemaAnalyst} is based on Korel's Alternating Variable Method (AVM)~\cite{Korel:AVM}.
This data generator uses a fitness function to evaluate how well the test data satisfies the
requirements, and guide the test data generator towards producing higher quality test data that
satisfies more of the requirements. The resulting test data is saved as a JUnit test suite.

\textit{SchemaAnalyst} also includes features to evaluate the quality of the generated test data. The
coverage of the generated test data is the percentage of test requirements satisfied by the test data. 
Test data quality can also be measured using the provided mutation testing tools. When executed in
mutation testing mode, \textit{SchemaAnalyst} will generate mutant database schemas and compare the
behavior of the test data on the original and mutant schemas. \textit{SchemaAnalyst} includes 14
different mutation operators that can be used to assess test suite quality.

\input{figures/sa}

\section{Implementation}
\textit{SchemaAnalyst} is implemented as a Java program.  Designed with extensibility in mind, the
\textit{SchemaAnalyst} tool is divided into 13 packages. The \texttt{sqlrepresentation} package
provides an intermediate Java representation of data structures in relational databases,
providing representations for database tables, columns, expressions, datatypes, integrity
constraints, and other relevant objects. A \texttt{sqlparser} package is provided that utilizes the
General SQL Parser~\cite{} to convert schemas expressed in the Structured Query Language (SQL) to the
Java intermediate representation. The \texttt{testgeneration} package provides a representation of test
suites and test cases, as well as test requirements and the nine provided coverage criteria. The
\texttt{data} package furnishes the three provided test data generators, as well as various generic
datatype representations for use in test data generation. The \texttt{dbms} package provides support
for the three currently supported DBMSs, and includes the classes that enable interaction with
installed DBMSs. The \texttt{sqlwriter} package provides support for creating SQL statements for use
with DBMSs, and is used with the \texttt{javawriter} package to encode the generated test data as a
JUnit test suite. The \texttt{mutation} package provides the mutation analysis functionality, 
including the 14 provided mutation operators, mutant equivalence and reduction features, and virtual
test suite executors.
\begin{enumerate}
\item design overview 
\item usage / demo
\item discussion of extensibility
\end{enumerate}

\section{Related Work}
\begin{enumerate}
\item canonical work
\item our past work
\item related work contrasted with ours
\end{enumerate}

\section{Conclusions and Future Work}\label{sec:conclusion}

Many data-centric services rely on the quality of the underlying data. Much of this data is managed by relational
databases, and the database schema protects the integrity of the data.  Testing the schema for correctness is vital to
ensuring data quality. \sa~is a tool that generates test data for a relational database schema, thereby increasing
confidence in the schema's correctness. Using a search-based technique, \sa~is able to provide high-quality test data
across multiple DBMSs. \sa~also includes an evaluation framework that provides \numprovidedschemas~case-study schemas
and support for efficient mutation analysis. In addition to being used in \numuniquepapers~published studies, the
presented tool is now available to the public on GitHub~\cite{tool}. With an open-source license and a modular design,
\sa~is an extensible tool for search-based test data generation and mutation testing, enabling the work of researchers
and practitioners.

In future work, we plan to evaluate how \sa~helps the people who design and test relational database schemas. We will
also integrate the tool with others that support software maintenance activities like regression
testing~\cite{Kapfhammer2008} and fault localization~\cite{Clark2011}. Next, we will further extend \sa~so that it
enables the testing of recent NoSQL systems. Ultimately, the current version of \sa, our planned extensions, and the
features and studies contributed by new users of this open-source tool will yield a comprehensive approach to testing
data-centric applications.


\bibliographystyle{IEEEtran}
\bibliography{IEEEabrv,bib/bibtex}

\end{document}


